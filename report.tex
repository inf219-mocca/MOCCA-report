\documentclass[12pt,a4paper,oneside,article]{memoir}

\usepackage{polyglossia}
\setdefaultlanguage{english}
\usepackage{fontspec}

\defaultfontfeatures{Ligatures=TeX}
\newfontfeature{Microtype}{protrusion=default;expansion=default}
\usepackage[final]{microtype}
\setmainfont{Linux Libertine O}
\setsansfont{Linux Biolinum O}
\setmonofont{DejaVu Sans Mono}

\usepackage{subfiles}
\usepackage{tabularx}
\usepackage{tabu}
\usepackage{booktabs}
\usepackage{multirow}
\usepackage{float}
\usepackage{amsmath,amsfonts,amssymb}
\usepackage{mathtools}
\usepackage[shortlabels]{enumitem}

\usepackage{tikz}
\usepackage[subpreambles=true]{standalone}
\usepackage{import}
\usepackage{graphicx}
\usepackage{hyperref}
\hypersetup{hidelinks}
\usepackage{xcolor, colortbl, array}

\usepackage{listings}
\usepackage{color}
\lstset{
  language=Python,
  frame=tb,
  breaklines=true,
  breakatwhitespace=true,
  keepspaces=true,
  columns=fullflexible,
  showspaces=false,
  showstringspaces=false,
  showtabs=false,
  basicstyle=\ttfamily\footnotesize
}

\usepackage[autostyle,strict,autopunct]{csquotes}
\usepackage[style=ieee,backend=biber]{biblatex}
\bibliography{bibliography}

\usepackage{chngcntr}
\counterwithin{table}{section}
\numberwithin{equation}{chapter}
\counterwithin{figure}{section}
\setenumerate[0]{label= (\alph*)}
\AtBeginDocument{\counterwithin{lstlisting}{section}}
\counterwithout{section}{chapter}
\newsubfloat{figure}

\chapterstyle{article}
\pagestyle{headings}

\title{MOCCA Operational Controller for Coffee Availability}
\author{Eivind Dagsland Halderaker \and Sondre Åsemoen Nilsen}
\date{Spring 2019}
\begin{document}
\begin{titlingpage}

\newcommand{\HRule}{\rule{\linewidth}{0.5mm}}
\centering

\textsc{\LARGE University of Bergen \\ Department of informatics}\\[1.5cm] %

\HRule\\[0.5cm]
\begin{Huge}
	\bfseries{\thetitle}\\[0.7cm]
\end{Huge}
\HRule\\[0.5cm]

{\large \emph{Author:} \theauthor}\\
{\large \emph{Supervisor:} Albin Severinson\\[2cm]}

\centerline{\includegraphics[scale=1.9]{figures/canvasWithFaculty}}
{\large \thedate}\\[3cm]
\vfill

\begin{abstract}
One of the challenges facing students at the informatics departmenet at the 
University of Bergen is knowing whether there is fresh coffee available in the 
study hall or not.  Students often lose valuable study time by walking over to 
the coffee maker, only to discover that the pot is empty or that the coffee is 
cold.

In this report, we present \textit{MOCCA} (MOCCA Operational Controller for
Coffee Availability), an internet-of-things coffee brewer designed to address
this issue. Specifically, \textit{MOCCA} enables students to visit a website and
immediately know the status of coffee. This is done by measuring the coffee's
temperature, brewer's power usage, and lastly the amount of coffee.
\end{abstract}
\end{titlingpage}

\clearpage

\chapter{Report}
\section*{Todos}
\begin{itemize}
\item List of acronyms/synonyms/glossary
\item Links/references to technologies/frameworks
\item Sensors requires proper names and so on
\item Give thanks to everyone who has helped
\end{itemize}

\section{Introduction}\label{sec:introduction}
At our study hall our most prized possestion is problably the coffee 
machine. It is a valuable piece of comfort that we have, and is used from eight 
in the morning until the last student goes home.

The problem is that you're never sure whether or not there is actually coffee
available, and whenever there is none you waste precious minutes in the couch
waiting for it to finish brewing. There have been many attempts at solving this
problem throughout the years, none of which have survived past the initial
testing phase. This project breaks the chain and have a working product that can
withstand the test of time.

In this report, we describe a system combining different sensors together with
an Arduino and a Raspberry Pi to monitor a drip coffee brewer and service an
API, we call this system \textit{MOCCA}. It's task is to measure both the
quality and quantity of coffee.

For quality we use both a temperature sensor and power current sensor (?). The
temperature is essential to high quality coffee, since you do not want it
chilled down and definitely not warmed up again. Another factor of quality is
the age of the brew. The good taste of coffee deteriorates as time goes on if it
has been held warm for some time. To measure the age of a brew we monitor the
amount of power the brewer is draining. A coffee brewer has essentially three
states: turned off, brewing, and warming. By saving a time stamp at the start of
a new brew, it is trivial to calculate age.

Quantity on the other hand is done by image processing. Images are taken of the
coffee pot and in turn measure with regards to the amount of dark pixels.
Taking the background of the pot into account, we produce an estimate of
the amount of coffee.

In section \ref{sec:system-overview}, we give an overview as well as a technical
description of the system and its various subsystems. In section
\ref{sec:conclusion} we give our concluding remarks as well as future work in
section \ref{sec:future-work}.

\section{System overview}\label{sec:system-overview}
\begin{figure}[h]
  \centerfloat{}
  \scalebox{.75}{\documentclass[border=3mm]{standalone}
\usepackage{tikz}
\usetikzlibrary{positioning,fit,arrows.meta,backgrounds}

\tikzset{
    module/.style={%
        draw, rounded corners,
        minimum width=#1,
        minimum height=7mm,
        font=\sffamily
        },
    module/.default=2cm,
    >=LaTeX
}

\begin{document}
\begin{tikzpicture}[->]
  \node[module] (Loop) {Main loop};
  % Arduino box
  \node[module, below left=1cm and 5mm of Loop] (Read) {Read from sensors};
  \node[module, below left=8mm and -1cm of Read] (Temp) {Temperature};
  \node[module, below right=8mm and -1cm of Read] (Power) {Power draw};
  \node[fit=(Read) (Temp) (Power), draw, label={Arduino}] (SensorBox) {};
  \draw[->] (Read)--(Temp);
  \draw[->] (Read)--(Power);
  % Raspberry Pi processing box
  \node[module, below right=1cm and 5mm of Loop] (Camera) {Take picture};
  \node[module, below=of Camera] (CamProcess) {Process};
  \node[fit=(Camera) (CamProcess), draw, label={Raspberry Pi}] (Pi2Box) {};
  \draw[->] (Camera)--(CamProcess);
  % Data collection box
  \node[fit=(SensorBox) (Pi2Box) (Loop), draw, inner sep=2mm, label={Data collection}] (DataCollectionBox) {};
  \draw[->] (Loop)--(SensorBox);
  \draw[->] (Loop)--(Pi2Box);

  % Data processing box
  \node[module, right=2cm of {Loop-|DataCollectionBox.east}] (AsyncLoop) {Async loop};
  \node[module, below=of AsyncLoop] (Data) {Data processing};
  \node[module, below=of Data] (Storage) {Storage};
  \node[fit={(AsyncLoop) (Data) (Storage)},
        draw, inner sep=2mm,
        label={Data processing}]
        (MainBox) {};
  \draw[->] (AsyncLoop)--(Data);
  \draw[->] (Data)--(Storage);

  % Connect main loop and async loop
  \draw[<->,dashed] (Loop)--(AsyncLoop);

  % Serve box
  \node[module, right=2cm of {AsyncLoop-|MainBox.east}] (Server) {Server};
  \node[module, below=of Server] (API) {API};
  \node[module, below=of API] (Website) {Website};
  \node[fit={(Server) (API) (Website)}, draw, inner sep=2mm, label={Serving}] (ServeBox) {};
  \draw[->] (Server)--(API);
  \path[] (Server) edge[bend left=52.5] node [left] {} (Website);
  \draw[->] (API)--(Website);
  \draw[->] (MainBox)--(ServeBox);
\end{tikzpicture}
\end{document}
%%% Local Variables:
%%% mode: latex
%%% TeX-master: t
%%% End:
}
  \caption{Architecture diagram}\label{fig:architecture}
\end{figure}
\textit{MOCCA} is composed of three subsystems that are responsible for
collecting sensor data, processing and storing this data and presenting the data
to the user, respectively. See figure \ref{fig:architecture}.

\subsection{Data collection}\label{sec:data-collection}
The first subsystem is the data collection system consisting of a temperature
sensor measuring the temperature of the coffee, a power sensor measuring power
draw to determine the state of the brewer and a camera pointed towards the
coffee for measuring liquid levels. The two first sensors are controlled by an
Arduino micro controller while the camera however is connected to the Raspberry
Pi itself.

\subsubsection{Arduino}\label{sec:arduino}
The code for the Arduino is very simple in all honesty, but it does its job. Any
code on an Arduino runs in a tight loop that continuously reads from its sensors
and saves the data to a queue. Since we only read the data from the Arduino
every 10 seconds we needed to device a way to get the latest data while ensuring
that we didn't read it in the middle of a write to the serial interface. Our
solution was to have the Arduino continuously read to the queue, discarding the
last item in it for every reading, while waiting for us to send a predetermined
message saying that we are ready for data. In our case this message is simply
one: \lstinline{1}.

Once the Arduino receives this it enters a write-mode --- in a sense --- and
fetches the latest reading from the queue, returns it, clears the queue and
exits the current state returning to reading data. The sensors themselves all
came with ready-made libraries that you simply need to download, import them and
then call the documented functions and you are good to go.

\subsubsection{Raspberry Pi}\label{sec:raspberry-pi}
Raspberry Pi is a small computer with the basics you need: USB, HDMI, WiFi, and,
most important for us, support for a camera. Though lacking in computational
power, its size and affordability makes it a perfect data cruncher for our case,
it can fit in a compact container near the drip coffee brewer. We used Raspberry
Pi's own camera module for ease of use. The camera itself is plug-and-play and
has more than high enough resolution for our use. We only need a rough estimate
of how full the coffee pot is and high resolution means more data needed to be
processed. Most of which is redundant.

\subsection{Data processing}\label{sec:data-processing}
The data processing subsystem is where the meat of the system lies, an
asynchronous task queue calls the data collection subsystem and once the data is
collected starts processing it. Every ten seconds an event fires which gathers
the data from the sensors and camera and processes the data given. A lot of the
heavy lifting here is done in NumPy, a high-performance library for scientific
computing with Python: once an image has been captured by our camera it is
transformed into a matrix by converting each pixel into a boolean value
depending on how bright the pixel is. Thanks to the excellent performance given
by NumPy even the limited computing power of the Raspberry Pi is no concern.

The temperature and power draw is simply read from the Arduino without any
processing necessary. This is not to say that we simply take the values and
insert them into our database, we still need to verify that the data that was
returned is valid data in our current state and what to do with it.

If our current assumed state is that the machine is keeping the coffee warm, a
reading where the temperature suddenly changes can for example be that someone
has taken away the carafe to fill a cup. These are just some of the scenarios
that the data processing layer needs to work around. The sensors are ---
essentially --- our interface to the real world, but the real world can be
deceitful and the data processing layer is where we tell truth from fiction.

\subsection{Data serving}\label{sec:data-serving}
The final system is the user facing system, where we serve the data that we have
gathered and processed. This subsystem consists of two parts, one back-end and
one front-end. The back-end serves the data that has been gathered and processed
via an API that is consumable by others that want to build on top of this, and
most importantly, our front-end.

\subsection{Back-end}\label{sec:back-end}
The back-end is written in Django, a high-level web framework for Python and is
definitely the most complex part of the project, in part because the real world
is messy, but also because you have to orchestrate a service that communicates
with the both the Arduino and takes a picture with our camera while at the same
time process and serve the data. Due to the constraints put on the system we
needed to run the data fetching in the background in an asynchronous task
runner as we could not have the user facing part of the system lock itself up
while it was processing the data.

As mentioned, the real world is messy --- there are a ton of moving parts,
variables and ``states'' it can be in. When reading from the sensors, how do you
know if the current ``state'' of the world is in order? For instance, the coffee
machine can be powered off by two different parts: its own power supply and the
socket itself as both of these run on separate timers. When getting the power
status from the Arduino, how do you ensure that the machine has been in a
consistent phase and the reading you are getting is not a one-time event? The
amount of possible things you have to account for is near endless, and we have
most definitely not been able to foresee or cover for all of them.

\subsection{Front-end}\label{sec:front-end}
\begin{figure}[ht]
  \centerfloat{}
  \subbottom[When the brewer has lost power]{
    \includegraphics[width=0.9\textwidth]{figures/nopower}
  }
  \subbottom[When it has been brewing, but lost power during]{
    \includegraphics[width=0.9\textwidth]{figures/brewing}
  }
  \caption{Examples of various states represented on the website}\label{fig:website}
\end{figure}
Though not nearly as complex as the back-end, the front-end still had its own
fair share of minor issues. We decided early on with using React due to its
popularity for web applications nowadays as a learning experience. It is a very
simple web application where users can visit in their web browser and instantly
see how the coffee is doing. Some examples of the core functionality can be seen
in figure \ref{fig:website}.

The web application is in many ways similar to the event loop in our back-end in
that it automatically attempts to fetch the latest data from the API every 10
seconds. Though we initially had some concerns about the load to the system we
have yet to see any performance issues, even while processing our data.

Once the uses visits the web application they are presented with a quick
overview of how the coffee looked at the latest reading. These states are
represented by an icon that fills up when the coffee is brewing, or a dead
coffee pot if the machine is without power and so on. For flavor some text
describing the data is added, if the coffee is nearly empty the user is told
that ``you better run'' to emphasize the low amount of coffee left. A history
page is also included if the user wants to see the trends for the coffee.

\section{Conclusion}\label{sec:conclusion}
We have created a system to somewhat successfully monitor the coffee in a drip
coffee brewer. Current and temperature sensors connected to an Arduino gather
information about the quality of the coffee. The temperature and current is then
sent to a Raspberry Pi via an USB serial interface. Using a camera connected to
the Raspberry Pi we capture an image of the coffee pot to process the amount of
liquid in the container. Processing this data in turn gives us an insight in the
quality and quality of the coffee.

A Django REST API then presents temperature, age and amount to the outside 
world enabling others to create their own interface and representation of what
the state of the coffee is.

\section{Future work}\label{sec:future-work}
Though absolutely a success in the authors opinions, there are still quite a few
things left that we'd like to implement and others that are
would-be-cool-to-have. The biggest, and probably most pressing item to do is
some sort of shell for the hardware. We had grand plans of creating a nice
3D-printed case for both the Raspberry Pi and Arduino, but time was not on our
side before the scheduled deadline. This is something that we'd like to be able
to do during the summer so that it is ready for the new students arriving this
fall.

The front-end is overly simple and could benefit from having more pages showing
historical data but the potential for how and what is represented is also vast.
When discussing the project with students while we are testing the system some
suggestions keeps reappearing:

\begin{itemize}
\item Adding a scanner for the student ID to track how much coffee each student
  drinks, but also tracking who makes the most coffee or who takes the last cup
  of coffee without making a new one.
\item Tracking how many liters of coffee is consumed and being able to
  extrapolate fun and interesting data from this. For example how much more
  coffee is consumed during the exam period versus a regular day.
\end{itemize}

\clearpage
\appendix
\chapter{Licenses}\label{sec:licenses}
Both the code in the back-end and front-end are licensed under the MIT License,
a very popular open-source license. The primary reason for choosing this was due
to the very permissive nature of the license, in essence all it requires is
preservation of copyright and license notices. For derivative works you may
freely license them under whatever terms you want and with or without the
aforementioned source code.

The code for the Arduino however required the usage of a library that was
licensed under the APGLv3 license. The family of GPL licenses are what many call
viral licenses, if you use a library in your program that is licensed under
these your program is in essence infected and you now have to license yours
under the same license as well.

\chapter{Parts}\label{sec:parts}
% Fix formating
Raspberry Pi Camera module: 
https://www.raspberrypi.org/products/camera-module-v2/

Temperature sensor: https://www.adafruit.com/product/1748

Current sensor: https://www.sparkfun.com/products/11005

Raspberry Pi

Arduino

\clearpage{}
\renewcommand*{\UrlFont}{\footnotesize\ttfamily}
\printbibliography{}

\end{document}
%%% Local Variables:
%%% mode: latex
%%% TeX-master: t
%%% End:
