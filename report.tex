\documentclass[12pt,a4paper,oneside,article]{memoir}

\usepackage{polyglossia}
\setdefaultlanguage{english}
\usepackage{fontspec}

\defaultfontfeatures{Ligatures=TeX}
\newfontfeature{Microtype}{protrusion=default;expansion=default}
\usepackage[final]{microtype}
\setmainfont{Linux Libertine O}
\setsansfont{Linux Biolinum O}
\setmonofont{DejaVu Sans Mono}

\usepackage{subfiles}
\usepackage{tabularx}
\usepackage{tabu}
\usepackage{booktabs}
\usepackage{multirow}
\usepackage{float}
\usepackage{amsmath,amsfonts,amssymb}
\usepackage{mathtools}
\usepackage[shortlabels]{enumitem}

\usepackage{tikz}
\usepackage[subpreambles=true]{standalone}
\usepackage{import}
\usepackage{graphicx}
\usepackage{hyperref}
\hypersetup{hidelinks}
\usepackage{xcolor, colortbl, array}

\usepackage{listings}
\usepackage{color}
\lstset{
  language=Python,
  frame=tb,
  breaklines=true,
  breakatwhitespace=true,
  keepspaces=true,
  columns=fullflexible,
  showspaces=false,
  showstringspaces=false,
  showtabs=false,
  basicstyle=\ttfamily\footnotesize
}

\usepackage[autostyle,strict,autopunct]{csquotes}
\usepackage[style=ieee,backend=biber]{biblatex}
\bibliography{bibliography}

\usepackage{chngcntr}
\counterwithin{table}{section}
\numberwithin{equation}{chapter}
\counterwithin{figure}{section}
\setenumerate[0]{label= (\alph*)}
\AtBeginDocument{\counterwithin{lstlisting}{section}}
\counterwithout{section}{chapter}
\newsubfloat{figure}

\chapterstyle{article}
\pagestyle{headings}

\title{MOCCA Operational Controller for Coffee Availability}
\author{Eivind Dagsland Halderaker \and Sondre Åsemoen Nilsen}
\date{INF219, Spring 2019}
\begin{document}
\begin{titlingpage}

\newcommand{\HRule}{\rule{\linewidth}{0.5mm}}
\centering

\textsc{\LARGE University of Bergen \\ Department of informatics}\\[1.5cm] %

\HRule\\[0.5cm]
\begin{Huge}
	\bfseries{\thetitle}\\[0.7cm]
\end{Huge}
\HRule\\[0.5cm]

{\large \theauthor}\\
{\large \emph{Supervisor:} Albin Severinson\\[2cm]}

\centerline{\includegraphics[scale=1.9]{figures/canvasWithFaculty}}
{\large \thedate}\\[3cm]
\vfill

\begin{abstract}
One of the challenges facing students at the informatics department at the 
University of Bergen is knowing whether there is fresh coffee available in the 
study hall or not.  Students often lose valuable study time by walking over to 
the coffee maker, only to discover that the pot is empty or that the coffee is 
cold. In this report, we present \textit{MOCCA} (MOCCA Operational Controller 
for Coffee Availability), an Internet of Things coffee brewer designed to 
address this issue. Specifically, \textit{MOCCA} enables students to visit a 
website and immediately know the amount of remaining coffee, its temperature, 
and its age.
\end{abstract}
\end{titlingpage}

\clearpage

\section{Introduction}\label{sec:introduction}
At our study hall our most prized possession is probably the coffee 
machine. It is a valuable piece of comfort that we have, and is used from eight 
in the morning until the last student goes home. The problem is that you are 
never sure whether or not there is actually coffee available, and whenever there 
is none you waste precious minutes in the couch waiting for it to finish 
brewing. There have been many attempts at solving this problem throughout the 
years, none of which have survived past the initial testing phase. This project 
breaks the chain and have a working product that can withstand the test of time.

In this report, we present \textit{MOCCA} (MOCCA Operational Controller for 
Coffee Availability), a system for measuring both the temperature, age, and 
quantity of coffee. It orchestrates a RESTful API, which is a stateless 
web service that allows clients to make HTTP request to a server via 
specific URLs. The server then sends a response back that containing the 
requested data. Our API gives information about the temperature of the coffee, 
how old the brew is, and an estimate of the amount of coffee remaining. In turn 
this API is then used to display this information on a website.

\textit{MOCCA} achieves this by combining different sensors together with an 
Arduino and a Raspberry Pi as data cruncher, specifically, a contact-less 
infrared temperature sensor, an electrical current sensor, and a camera. 
Measuring temperature is fairly straight-forward with a infrared sensor. 
However, figuring out how old a brew is is a bit trickier. To measure the age of 
a brew we monitor the amount of current the brewer is drawing. A coffee brewer 
has essentially three states: turned off, brewing, and warming. By saving time 
stamps at the start of a new brew, it is trivial to calculate age at a given 
point. Quantity, on the other hand, is measured by image processing. Images are 
taken of the coffee pot and we estimate the amount of coffee remaining by 
counting the number of dark pixels,while accounting for the background of the 
pot.

This approach has the benefit of giving the students near our study hall 
real-time information about the drip brewer. Students are still able to use the 
brewer as usual without any interference from our product. Being non-intrusive 
is one of our goals as we do not want to sacrifice the coffee brewer's ease 
of use.
%The temperature is essential to high quality coffee, since you do not want it 
%chilled down and definitely not warmed up again. Another factor of quality is 
%the age of the brew. The good taste of coffee deteriorates as time goes on if 
%it has been held warm for some time.

The remainder of the rapport is organized as follows. Section 
\ref{sec:system-overview} is an overview as well as a technical
description of the system and its various subsystems. In Section
\ref{sec:conclusion}, we give our concluding remarks. Finally in Section 
\ref{sec:future-work}, and ending with acknowledgements in section 
\ref{sec:acknowledgements}.

\section{System overview}\label{sec:system-overview}
\begin{figure}[h]
  \centerfloat{}
  \scalebox{.75}{\documentclass[border=3mm]{standalone}
\usepackage{tikz}
\usetikzlibrary{positioning,fit,arrows.meta,backgrounds}

\tikzset{
    module/.style={%
        draw, rounded corners,
        minimum width=#1,
        minimum height=7mm,
        font=\sffamily
        },
    module/.default=2cm,
    >=LaTeX
}

\begin{document}
\begin{tikzpicture}[->]
  \node[module] (Loop) {Main loop};
  % Arduino box
  \node[module, below left=1cm and 5mm of Loop] (Read) {Read from sensors};
  \node[module, below left=8mm and -1cm of Read] (Temp) {Temperature};
  \node[module, below right=8mm and -1cm of Read] (Power) {Power draw};
  \node[fit=(Read) (Temp) (Power), draw, label={Arduino}] (SensorBox) {};
  \draw[->] (Read)--(Temp);
  \draw[->] (Read)--(Power);
  % Raspberry Pi processing box
  \node[module, below right=1cm and 5mm of Loop] (Camera) {Take picture};
  \node[module, below=of Camera] (CamProcess) {Process};
  \node[fit=(Camera) (CamProcess), draw, label={Raspberry Pi}] (Pi2Box) {};
  \draw[->] (Camera)--(CamProcess);
  % Data collection box
  \node[fit=(SensorBox) (Pi2Box) (Loop), draw, inner sep=2mm, label={Data collection}] (DataCollectionBox) {};
  \draw[->] (Loop)--(SensorBox);
  \draw[->] (Loop)--(Pi2Box);

  % Data processing box
  \node[module, right=2cm of {Loop-|DataCollectionBox.east}] (AsyncLoop) {Async loop};
  \node[module, below=of AsyncLoop] (Data) {Data processing};
  \node[module, below=of Data] (Storage) {Storage};
  \node[fit={(AsyncLoop) (Data) (Storage)},
        draw, inner sep=2mm,
        label={Data processing}]
        (MainBox) {};
  \draw[->] (AsyncLoop)--(Data);
  \draw[->] (Data)--(Storage);

  % Connect main loop and async loop
  \draw[<->,dashed] (Loop)--(AsyncLoop);

  % Serve box
  \node[module, right=2cm of {AsyncLoop-|MainBox.east}] (Server) {Server};
  \node[module, below=of Server] (API) {API};
  \node[module, below=of API] (Website) {Website};
  \node[fit={(Server) (API) (Website)}, draw, inner sep=2mm, label={Serving}] (ServeBox) {};
  \draw[->] (Server)--(API);
  \path[] (Server) edge[bend left=52.5] node [left] {} (Website);
  \draw[->] (API)--(Website);
  \draw[->] (MainBox)--(ServeBox);
\end{tikzpicture}
\end{document}
%%% Local Variables:
%%% mode: latex
%%% TeX-master: t
%%% End:
}
  \caption{System architecture. MORE DETAILS}\label{fig:architecture}
\end{figure}
\textit{MOCCA} is composed of three subsystems that are responsible for
collecting sensor data, processing and storing this data and presenting the data
to the user, respectively. See figure \ref{fig:architecture}.

\subsection{Data collection}\label{sec:data-collection}
The first subsystem is the data collection system. It consists of an 
infrared sensor measuring the temperature of the coffee, a current sensor 
measuring current to determine the state of the brewer and a camera pointed 
towards the coffee for measuring amount of coffee. The two first sensors are 
controlled by the Arduino micro controller while the camera is connected to the 
Raspberry Pi.

\subsubsection{Arduino}\label{sec:arduino}
Code on an Arduino usually runs in a loop that continuously reads from 
its sensors and saves the data to a queue. We read the data from the 
Arduino every 10 seconds so we needed to device a way to get the latest data 
while ensuring that we did not read it in the middle of a write to the serial 
interface. Our solution is for the Arduino to continuously read to the queue, 
discarding the last item in it for every reading, while waiting for the 
Raspberry Pi to send a predetermined message saying that it is ready for 
data. Once the Arduino receives this message enters a write-mode and fetches 
the latest reading from the queue, returns it, clears the queue and exits the 
current state returning to reading data. 

\subsubsection{Raspberry Pi}\label{sec:raspberry-pi}
The Raspberry Pi is a small computer equipped with e.g., USB, HDMI, WiFi, 
and, most important for us, support for a camera. Though lacking in 
computational power, its size and affordability makes it a perfect data 
cruncher for our case as it can fit in a compact container near the drip 
coffee brewer. We used the Raspberry Pi Foundation's own camera module and the 
camera itself is plug-and-play while still having more than high enough 
resolution for our use. We only need a rough estimate of how full the coffee pot 
is and high resolution means more data to be processed, most of which is 
redundant.

\subsection{Data processing}\label{sec:data-processing}
The data processing subsystem is where the meat of the system lies. An 
asynchronous task queue calls the data collection subsystem and once the data is
collected starts processing it. Every ten seconds an event fires which gathers
the data from the sensors and camera and processes the data given. A lot of the
heavy lifting here is done in NumPy, a high-performance library for scientific
computing with Python: once an image has been captured by our camera it is
transformed into a matrix by converting each pixel into a boolean value
depending on how bright the pixel is. Thanks to the excellent performance given
by NumPy, even the limited computing power of the Raspberry Pi is no concern.

The temperature and power draw is simply read from the Arduino without any
processing necessary. This is not to say that we simply take the values and
insert them into our database, we still need to verify that the data that was
returned is valid data in our current state and what to do with it.

If our current assumed state is that the machine is keeping the coffee warm, a
reading where the temperature suddenly changes can for example be that someone
has taken away the carafe to fill a cup. These are just some of the scenarios
that the data processing layer needs to work around. The sensors are ---
essentially --- our interface to the real world, but the real world does not 
always behave as we think it will. For example, noise from the sensors will 
happen, and our solution is to discard the data from the sensors.

\subsection{Data serving}\label{sec:data-serving}
The final subsystem is the user facing system, where we serve the data that we 
have gathered and processed. This subsystem consists of two parts, one back-end 
and one front-end. The back-end serves the data that has been gathered and 
processed via an API that is consumable by others that want to build on top of 
this, and most importantly, our front-end, the web application.

\subsection{Back-end}\label{sec:back-end}
The back-end is written in Django, a high-level web framework for Python and is
definitely the most complex part of the project. In part because the system can 
drift out of sync with the real world. Also because you have to orchestrate a 
service that communicates with both the Arduino and takes a picture with our 
camera while at the same time processing and serving the data. Due to the 
constraints put on the system we need to run the data fetching in the background 
in an asynchronous task runner as we could not have the user facing part of the 
system lock itself up while it was processing the data.

(...) here are a ton of moving parts, variables and states it can be in. When 
reading from the sensors, how do you know if the current state of the world is 
in order? For instance, the coffee machine can be powered off by two different 
parts: its own power supply and the socket itself as both of these run on 
separate timers. When getting the power status from the Arduino, how do you 
ensure that the machine has been in a consistent phase and the reading you are 
getting is not a one-time event? The amount of possible things you have to 
account for is near endless, and we have most definitely not been able to 
foresee or cover for all of them.

\subsection{Front-end}\label{sec:front-end}
\begin{figure}[ht]
  \centerfloat{}
  \subbottom[When the brewer has lost power.]{
    \includegraphics[width=0.9\textwidth]{figures/nopower}
  }
  \subbottom[When it has been brewing, but lost power during brewing.]{
    \includegraphics[width=0.9\textwidth]{figures/brewing}
  }
  \caption{Examples of various states represented on the website}\label{fig:website}
\end{figure}
The front-end consists of a web application that users can visit in their web 
browser and instantly see how the coffee is doing. We decided early on using 
React due to its popularity for web applications nowadays as a learning 
experience. Some examples of the core functionality can be seen in figure 
\ref{fig:website}. The web application is in many ways similar to the event 
loop in our back-end in that it automatically attempts to fetch the latest data 
from the API every 10 seconds.

When a user visits the web application he/she is presented with a quick
overview of how the coffee looked at the latest reading. These states are
represented by an icon that fills up when the coffee is brewing, or a dead
coffee pot if the machine is without power, with text describing the data. 
For example, if the coffee is nearly empty the user is told that ``you better 
run'' to emphasize the low amount of coffee left. A history page is also 
included if the user wants to see the trends for the coffee. (TODO: Define 
history page)

\section{Conclusion}\label{sec:conclusion}
We have created a system to successfully monitor the coffee in a drip coffee 
brewer. Current and infrared temperature sensors connected to an Arduino 
gather information about the age and temperature of the coffee. The 
temperature and current is then sent to a Raspberry Pi via an USB serial
interface. Using a camera connected to the Raspberry Pi, we capture images of 
the coffee pot to measure the amount of coffee in the container. Processing 
this data in turn gives us an insight in the temperature, age, and quantity of 
the coffee. These sensor are all non-intrusive, as to not discourage the 
students from brewing and drinking that sweet coffee. A Django REST API then 
presents temperature, age and amount to the outside world enabling others to 
create their own interface and representation of what the state of the coffee 
is. For others we have created a website to make it accessible for the masses.
 
\section{Future work}\label{sec:future-work}
There are still quite a few features left that we would like to implement and 
others that would-be-cool-to-have. The biggest, and probably most pressing 
item to do is some sort of casing for the hardware. We had grand plans of 
creating a nice 3D-printed case for both the Raspberry Pi and Arduino, but time 
was not on our side before the scheduled deadline. This is something that we 
would like to be able to do during the summer so that it is ready for the new 
students arriving this fall.

The front-end is simple and could benefit from having more pages showing
historical data. The potential for how and what is represented is vast. When 
discussing the project with students while we are testing the system some
suggestions keeps reappearing:
(TODO: Rewrite these parts)
Adding a scanner for the student ID to track how much coffee each student 
drinks, but also tracking who makes the most coffee or who takes the last cup 
of coffee without making a new one.
Tracking how many liters of coffee is consumed and being able to
extract fun and interesting data from this. For example, how much more
coffee is consumed during the exam period compared to a regular day.


\section{Acknowledgements}\label{sec:acknowledgements}
We would like to thank our supervisor Albin Severinson for giving us helpful, 
and rapid feedback during our almost weekly meetings, as well as Hjalmar 
Svenstrup Andersen and Sondre Vestad for consulting us on the matters of 
electronics and hardware.

\clearpage
\appendix
\chapter{Licenses}\label{sec:licenses}
Both the code in the back-end and front-end are licensed under the MIT License,
a very popular open-source license. The primary reason for choosing this was due
to the very permissive nature of the license, in essence all it requires is
preservation of copyright and license notices. For derivative works you may
freely license them under whatever terms you want and with or without the
aforementioned source code.

The code for the Arduino however required the usage of a library that was
licensed under the APGLv3 license. The family of GPL licenses are what many call
viral licenses, if you use a library in your program that is licensed under
these your program is in essence infected and you now have to license yours
under the same license as well. Hence, this code is licensed under GPL.

\chapter{Parts}\label{sec:parts}
% Fix formating
\begin{itemize}
\item \textbf{Raspberry Pi Camera Module v2.0} \newline
Manufacturer: Raspberry Pi Foundation

\item \textbf{Contact-less Infrared Temperature Sensor} \newline
Manufacturer: Melexis \newline
Part no. MLX90614

\item \textbf{Non-invasive Current Sensor} \newline
Manufacturer: EChun Electronic \newline
Part no. ECS1030-L72

\item \textbf{Raspberry Pi 3 Model B+} \newline
Manufacturer: Raspberry Pi Foundation 

\item \textbf{Arduino Uno Rev3} \newline
Manufacturer: Arduino

\end{itemize}

\chapter{API Reference}\label{sec:api-reference}

\clearpage{}
\renewcommand*{\UrlFont}{\footnotesize\ttfamily}
\printbibliography{}

\end{document}
%%% Local Variables:
%%% mode: latex
%%% TeX-master: t
%%% End:
