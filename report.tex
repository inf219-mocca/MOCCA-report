\documentclass[12pt,a4paper,oneside,article]{memoir}

\usepackage{polyglossia}
\setdefaultlanguage{english}
\usepackage{fontspec}

\defaultfontfeatures{Ligatures=TeX}
\newfontfeature{Microtype}{protrusion=default;expansion=default}
\usepackage[final]{microtype}
\setmainfont{Linux Libertine O}
\setsansfont{Linux Biolinum O}
\setmonofont{DejaVu Sans Mono}

\usepackage{subfiles}
\usepackage{tabularx}
\usepackage{tabu}
\usepackage{booktabs}
\usepackage{multirow}
\usepackage{float}
\usepackage{amsmath,amsfonts,amssymb}
\usepackage{mathtools}
\usepackage[shortlabels]{enumitem}

\usepackage{tikz}
\usepackage{graphicx}
\usepackage{hyperref}
\hypersetup{hidelinks}
\usepackage{xcolor, colortbl, array}

\usepackage{listings}
\usepackage{color}
\lstset{
  language=Python,
  frame=tb,
  breaklines=true,
  breakatwhitespace=true,
  keepspaces=true,
  columns=fullflexible,
  showspaces=false,
  showstringspaces=false,
  showtabs=false,
  basicstyle=\ttfamily\footnotesize
}

\usepackage[autostyle,strict,autopunct]{csquotes}
\usepackage[style=ieee,backend=biber]{biblatex}
\bibliography{bibliography}

\usepackage{chngcntr}
\counterwithin{table}{section}
\numberwithin{equation}{chapter}
\counterwithin{figure}{section}
\setenumerate[0]{label= (\alph*)}
\AtBeginDocument{\counterwithin{lstlisting}{section}}
\counterwithout{section}{chapter}

\chapterstyle{hangnum}
\pagestyle{ruled}

\title{MOCCA: MOCCA Operational Controller for Coffee Availability}
\author{Eivind D. Halderaker, Sondre Å. Nilsen}
\date{Spring 2019}
\begin{document}

\maketitle
\tableofcontents

\begin{abstract}
  As students with access to free coffee in our study hall our biggest problem
  is never knowing whether there will be fresh coffee available from it in times
  of need. \textit{MOCCA} is an attempt to address and fix this problem once and
  for all: a single source of truth about the availability of fresh coffee.
\end{abstract}

\section{Todos}
\begin{itemize}
\item List of acronyms/synonyms/glossary
\item Links/references to technologies/frameworks
\item Sensors requires proper names and so on
\item Give thanks to everyone who has helped
\end{itemize}

\section{Introduction}\label{sec:introduction}
Introductions really are the hardest part about writing anything, so with that
out of the way, lets talk about \textit{MOCCA}. It's an attempt to both write
and wire together a solution that combines hardware and software to solve a long
standing problem we've had in our study hall. The coffee machine is probably the
most prized and valuable piece of comfort that we have, and is used from eight
in the morning until the last student goes home.

The great problem is that you're never sure whether or not there is actually
coffee available, and whenever there is none you waste precious minutes in the
couch waiting for it to finish brewing. There have been many attempts at solving
this problem throughout the years, none of which have survived past the initial
testing phase. This project aims to break the chain and have a working product
that can withstand the test of time.

\section{Technical details}\label{sec:technical-details}
\textit{MOCCA} is written in a wide range of languages and technologies, the
back-end is written in Python using the Django framework while the front-end is
written in React, while communication between these two happens across a REST
API. The most important piece of software and hardware however is the Arduino
that is --- essentially --- our interface between the real world and our
software. It reads the temperature of the coffee, how much power the machine is
drawing and instructs our camera when to take a photo.

\subsection{Why Python and Django}\label{sec:why-python-django}
For the back-end Python was chosen specifically for its breadth of available
libraries and solutions, and ease of use. We briefly talked about other
languages but Python was a very clear winner. The choice of Django however was a
bit more contentious, in the very beginning the battle plan was to use Flask
which would've meant that we'd have to do a lot of the heavy lifting ourselves.
After a fairly quick trial period where we developed a few prototypes the choice
fell on Django, mostly due to the batteries-included approach of it but perhaps
mostly due to Sondre's familiarity with it given he had previous experience with
it.

Thanks to Django we were able to build on top of existing battle-tested
libraries, we used Django REST Framework for the heavy lifting of implementing
the API, we used drf-yasg (Django REST Framework - Yet Another Swagger
Generator) for creating Swagger documentation of the API. For communicating with
our Arduino we used Pyserial and for asynchronously polling it we used Celery.

\subsection{Licenses}\label{sec:licenses}
Both the code in the back-end and front-end are licensed under the MIT License,
a very popular open-source license. The primary reason for choosing this was due
to the very permissive nature of the license, in essence all it requires is
preservation of copyright and license notices. For derivative works you may
freely license them under whatever terms you want and with or without the
aforementioned source code.

The code for the Arduino however required the usage of a library that was
licensed under the APGLv3 license. The family of GPL licenses are what many call
viral licenses, if you use a library in your program that is licensed under
these your program is in essence infected and you now have to license yours
under the same license as well.

\subsection{Implementation}\label{sec:implementation}
\subsubsection{Arduino}\label{sec:arduino}
The code for the Arduino is very simply in all honesty, but it does the job
required. Any code on an Arduino runs in a tight loop that continuously reads
from its sensors and saves the data to a queue. Since we read from the Arduino
every ten seconds, we only want the latest data but we don't want to block
getting the data on the Arduino itself reading its sensors. The way we worked
around this issue was to have the Arduino continuously read to the queue,
discarding the last item in it for every reading, while waiting for us to send a
predetermined message saying that we are ready for data. In our case this
message is simply ``1''.

Once the Arduino receives this it enters a write-mode --- in a sense --- and
fetches the latest reading from the queue, returns it, clears the queue and
exits the current state returning to reading data. The sensors themselves all
came with ready-made libraries that you simply need to download, import them and
then call the documented functions and you are good to go.

\subsubsection{Back-end}\label{sec:back-end}
The back-end is definitely the most complex part of the project, in part because
the real world is messy, but also because you have to orchestra a service that
communicates with the Arduino while also serving an API, which required us to do
the data fetching in the background in a asynchronous task runner.

As mentioned, the real world is messy --- there are a ton of moving parts,
variables and ``states'' it can be in. When reading from the sensors, how do you
know if the current ``state'' of the world is in order? For instance, the coffee
machine can be powered off by two different parts: its own power supply, but
also the power plug itself as it runs on a timer. When getting the power status
from the Arduino, how do you ensure that the machine has been in a consistent
phase and the reading you are getting is not a one-time event? The amount of
possible things you have to account for is near endless, and we have most
definitely not been able to foresee or cover for all of them.

\subsubsection{Front-end}\label{sec:front-end}
The hardest part of the front-end was definitely the CSS, and this only partly
in jest. Though the least complex of the two -ends, it still had its own fair
share of minor issues. We decided early on with using React due to its
popularity for web applications nowadays as a learning experience. React is in
many ways the exact opposite of Django, no batteries included but the ecosystem
for it is vast. We were again able to lean on existing battle-tested solutions,
using React Router for creating pages, Axios for querying the API and moment for
parsing dates.

The front-end does a GET request to the API endpoint that our back-end serves
every ten seconds and tries to display the current status of the coffee. You
have two possible views, a simple ``the coffee is hot, go get it!'' page and a
history page listing the last 25 readings.

\section{Conclusion}\label{sec:conclusion}
Though absolutely a success in the authors opinions, there are still quite a few
things left that we'd like to implement and others that are
would-be-cool-to-have. We have talked about doing statistics about which
students drink the most coffee, who is the best at starting a new pot of coffee
if the old one is empty and so on.

The biggest, and probably most pressing item to do is some sort of shell for the
hardware. We had grand plans of creating a nice 3D-printed case for both the
Raspberry Pi and Arduino, but time was not on our side before the scheduled
deadline. This is something that we'd like to be able to do during the summer so
that it is ready for the new students arriving this fall.

\clearpage{}
\renewcommand*{\UrlFont}{\footnotesize\ttfamily}
\printbibliography{}

\end{document}
%%% Local Variables:
%%% mode: latex
%%% TeX-master: t
%%% End:
